\documentclass{article}
\usepackage{graphicx}
%%%%%%%%%%%%%%%%%%%%%%%%%%%%%%%%%%%%%%%%
\usepackage{amsmath}
\usepackage{amsthm}
%%%%%%%%%%%%%%%%%%%%%%%%%%%%%%%%%%%%%%%%
\usepackage{tikz}
\usetikzlibrary{chains,shapes.multipart}
\usetikzlibrary{shapes,calc}
\usetikzlibrary{automata,positioning}
% در این فایل، دستورها و تنظیمات مورد نیاز، آورده شده است

\begin{document} 
	\begin{figure}[!htbp]
	\centering
	\begin{tikzpicture}[start chain=going right,>=latex,node distance=0pt]
	\node[draw,rectangle,on chain,minimum size=1.5cm] (rr) {};
	\node[draw,rectangle,on chain,draw=white,minimum size=1.3cm]{};
	% the rectangular shape with vertical lines
	\node[rectangle split, rectangle split parts=6,
	draw, rectangle split horizontal,text height=1cm,text depth=0.5cm,on chain,inner ysep=0pt] (wa) {};
	\fill[white] ([xshift=-\pgflinewidth,yshift=-\pgflinewidth]wa.north west) rectangle ([xshift=-15pt,yshift=\pgflinewidth]wa.south);
	\node at (wa.east) (A){};
	\draw [-latex] (A) --+(30:1.5) coordinate (B1);
	\draw [-latex] (A) --+(-30:1.5) coordinate (B2);
	% the circle
	\node [draw,circle,on chain,minimum size=1cm] at (B1) (se1) {$k_1$};
	\node [draw,circle,on chain,minimum size=1cm] at (B2) (se2) {$k_2$};
	\draw [-latex] (se1.east) --+(-25:1.65) coordinate (C1);
	\draw [-latex] (se2.east) --+(25:1.65) coordinate (C2);
	\node (O) at ($(C1)!0.5!(C2)$) {};
	\node [draw,circle,on chain,minimum size=3pt] at (O) (C3){};
	\draw [-latex] (C3)--+(0:2)node[right] {$\mu$};
	% the arrows and labels
	%   \draw[->] (se.east) -- +(20pt,0) node[right] {$\mu$};
	\draw[<-] (wa.west) -- +(-20pt,0) node[left] {$\lambda$};
	\node[align=center,below] at (rr.south) {بالقوه مشتریان جمعیت};
	\node[align=center,below] at (wa.south) {صف تشکیل};
	\node[align=center,below] at (se2.south) {سرویس‌دهی خطوط};
	
	\end{tikzpicture}
	\caption{اجزای یک سیستم صف}
	\label{fig:queue}
\end{figure}
\end{document}